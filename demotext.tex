Dies hier ist ein \textsf{Blindtext} zum Testen von
Textausgaben. Wer diesen Text liest, ist selbst
schuld: $\sin^2x+\cos^2y=1$. Der Text gibt lediglich den Grauwert
der Schrift an mit $E=mc^2$. \textbf{Ist das wirklich so?} Ist es
gleichgültig, ob ich schreibe: »Dies ist ein
\textsf{Blindtext}« oder »\textsf{\textbf{Huardest gefburn}«? Kjift} --~mitnichten! 
Ein \textsf{Blindtext} bietet mir wichtige
Informationen: $\sqrt[n]{a}\cdot\sqrt[n]{b}=\sqrt[n]{ab}$. An ihm messe ich die
\emph{Lesbarkeit einer Schrift}, ihre Anmutung, wie
harmonisch die \textsc{Figuren zueinander stehen} und prüfe, wie breit
oder schmal sie läuft. $\frac{\sqrt[n]{a}}{\sqrt[n]{b}}=\sqrt[n]{\frac{a}{b}}$ Ein \textsf{Blindtext} sollte
möglichst \textit{\bfseries viele verschiedene Buchstaben} enthalten
und in der Originalsprache gesetzt sein: $\mathrm{d}\Omega=\sin\delta\mathrm{d}\delta\mathrm{d}\phi$ Er
muss keinen Sinn ergeben, sollte aber lesbar
sein. {\sffamily \textbf{\textsc{Fremdsprachige}} Texte wie »\textit{Lorem
ipsum}« dienen nicht dem eigentlichen Zweck, da sie eine falsche
Anmutung vermitteln}. \texttt{\textbf{TT-Fett}}, \texttt{\textit{TT-Kursiv}}
und \texttt{\textbf{\textit{TT-Fett-Kursiv}}}, soweit möglich.\par
Siehe auch \texttt{http://www.blindtextgenerator.de}.


