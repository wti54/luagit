% %%%%%%%%%%%%%%%%%%%%%%%%%%%%%%%%%%%%
%   Grundtext in A5
%.  Anführungszeichen funktionieren
%
%%%%%%%%%%%%%%%%%%%%%%%%%%%%%%%%%%%%%
\documentclass[a5paper,english, ngerman,parskip=half-]{scrartcl}
%\usepackage{libertinus}
%\usepackage[T1]{fontenc}
\usepackage{fontspec}
\usepackage{libertinus}
\usepackage{microtype}
\usepackage{geometry}
\usepackage{babel}
\babelprovide[hyphenrules=ngerman-x-latest]{ngerman}
% \babelprovide ist optional. Man bekommt aber
% weniger Trennfehler, weit unter 0.1%
\usepackage[colorlinks]{hyperref} %soll als letztes Paket gestartet werden
\hypersetup{linkcolor=blue}
\usepackage[autostyle]{csquotes}
\usepackage{eurosym}
\usepackage{blindtext}

\title{Textsatz mit \LaTeX}
\author{Dr. Wolfgang Tischendorf}
\date{\today}
\begin{document}
\maketitle
\tableofcontents
\foreignlanguage{english}{%
\begin{abstract}
This is a short introduction into
the professional typesetting system \TeX.
\end{abstract}}
\section{Probeteil 1}
Erste Versuche mit dem Setzen
eines einem \LaTeX-Dokuments
und der Schrift Libertinus.
Probiert werden die Umlaute ä, ö und ü sowie das ß. Wichtig sind die  \enquote{Anführungszeichen}. \par
Hier noch einmal geschriebene „Anführungszeichen” = Shift + Alt + w und Shift + Alt + 2 \par
Und die kleinere ‚Version‘ = Alt + s und Alt + \# \par
Oder doppelte »Guillemets« = Shift + Alt + q und Alt + q  \par
Oder einfache ›Guillemets‹ = Shift + Alt + n und Shift + Alt + b \par
\section{Probeteil 2}
Erste Versuche mit dem Setzen
eines \LaTeX"=Dokuments für
0,--\,\euro.
\blindtext[2]
\section{Probeteil 3}
Dies hier ist ein \textsf{Blindtext} zum Testen von
Textausgaben. Wer diesen Text liest, ist selbst
schuld: $\sin^2x+\cos^2y=1$. Der Text gibt lediglich den Grauwert
der Schrift an mit $E=mc^2$. \textbf{Ist das wirklich so?} Ist es
gleichgültig, ob ich schreibe: »Dies ist ein
\textsf{Blindtext}« oder »\textsf{\textbf{Huardest gefburn}«? Kjift} --~mitnichten! 
Ein \textsf{Blindtext} bietet mir wichtige
Informationen: $\sqrt[n]{a}\cdot\sqrt[n]{b}=\sqrt[n]{ab}$. An ihm messe ich die
\emph{Lesbarkeit einer Schrift}, ihre Anmutung, wie
harmonisch die \textsc{Figuren zueinander stehen} und prüfe, wie breit
oder schmal sie läuft. $\frac{\sqrt[n]{a}}{\sqrt[n]{b}}=\sqrt[n]{\frac{a}{b}}$ Ein \textsf{Blindtext} sollte
möglichst \textit{\bfseries viele verschiedene Buchstaben} enthalten
und in der Originalsprache gesetzt sein: $\mathrm{d}\Omega=\sin\delta\mathrm{d}\delta\mathrm{d}\phi$ Er
muss keinen Sinn ergeben, sollte aber lesbar
sein. {\sffamily \textbf{\textsc{Fremdsprachige}} Texte wie »\textit{Lorem
ipsum}« dienen nicht dem eigentlichen Zweck, da sie eine falsche
Anmutung vermitteln}. \texttt{\textbf{TT-Fett}}, \texttt{\textit{TT-Kursiv}}
und \texttt{\textbf{\textit{TT-Fett-Kursiv}}}, soweit möglich.\par
Siehe auch \texttt{http://www.blindtextgenerator.de}.



\section{Test 1}
\end{document}
